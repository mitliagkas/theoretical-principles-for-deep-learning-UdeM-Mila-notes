\documentclass{article}

%----------------------------------------
% Packages
%----------------------------------------
\usepackage[left=1in, right=1in, top=1in, bottom=1in]{geometry}
\usepackage{graphicx}
\usepackage{amsmath,amsbsy,amssymb,amsfonts,amsthm}
\usepackage{nicefrac}
\usepackage{mathtools}
\usepackage{color}
\usepackage{xspace} % Correct macro spacing
\usepackage[numbers]{natbib} % For citations
\usepackage{times}
\usepackage{booktabs}
\usepackage{lscape}
\usepackage{graphicx,subfigure}
%\usepackage[small,bf]{caption}
\usepackage{algorithm,algorithmic} 
\usepackage{hyperref}

\usepackage{shadethm}
\usepackage{fancyhdr}
\pagestyle{fancy}
\lhead{This is my name}
\rhead{this is page \thepage}
\usepackage[table,xcdraw]{xcolor}
%\usepackage{pgfplots}
\usepackage{tabularx}
\usepackage{fancyhdr}
\pagestyle{fancy}
\lhead{IFT 6085 - Theoretical principles for deep learning}
\rhead{Lecture 3: January 15, 2020}


\newshadetheorem{thm}{Theorem}
\newshadetheorem{defn}[thm]{Definition}
\newshadetheorem{assm}[thm]{Assumption}
\newshadetheorem{prop}[thm]{Property}
\newshadetheorem{eg}[thm]{Example}
\newshadetheorem{lemma}[thm]{Lemma}


\usepackage{amsmath}

\usepackage{import}
\import{../utils/}{macros.tex}

\DeclareMathOperator*{\argmax}{\arg\max}
\DeclareMathOperator*{\argmin}{\arg\min}
\DeclareMathOperator*{\E}{\mathbb{E}}
\DeclareMathOperator*{\I}{\mathbb{I}}
\DeclareMathOperator*{\V}{\mathbb{V}}

\newcommand{\diag}[0]{\operatorname{diag}}
\newcommand{\vect}[1]{\mathbf{#1}}
\newcommand{\vects}[1]{\boldsymbol{#1}}
\newcommand{\matr}[1]{\mathbf{#1}}
\newcommand{\matrs}[1]{\boldsymbol{#1}}

\renewcommand{\P}[0]{\mathbb{P}}
\newcommand{\Q}[0]{\mathbb{Q}}

\DeclareMathOperator*{\tr}{tr}
\DeclareMathOperator*{\prox}{prox}
\DeclareMathOperator*{\conv}{conv}
%\DeclareMathOperator*{\dom}{dom}
\DeclareMathOperator*{\minimize}{minimize}
\DeclareMathOperator*{\maximize}{maximize}
%\DeclareMathOperator*{\sign}{sign}
\DeclareMathOperator*{\vecop}{vec}
\DeclareMathOperator*{\Poisson}{Poisson}
\DeclareMathOperator*{\Cat}{Cat}
\DeclareMathOperator*{\Dir}{Dir}
\DeclareMathOperator*{\Exp}{Exp}
\DeclareMathOperator*{\DiscreteUniform}{DiscreteUniform}
\DeclareMathOperator*{\Uniform}{Uniform}
\newcommand{\R}{\mathbb{R}}
\newcommand{\N}{\mathbb{N}}
\newcommand{\X}{\mathcal{X}}

%\DeclarePairedDelimiter{\abs}{\lvert}{\rvert}
%\DeclarePairedDelimiter{\norm}{\lVert}{\rVert}
\DeclarePairedDelimiter{\inprod}{\langle}{\rangle}
\DeclarePairedDelimiter{\biginprod}{\big\langle}{\big\rangle}
\DeclarePairedDelimiter{\Biginprod}{\Big\langle}{\Big\rangle}
\DeclarePairedDelimiter{\bigginprod}{\bigg\langle}{\bigg\rangle}
\DeclarePairedDelimiter{\Bigginprod}{\Bigg\langle}{\Bigg\rangle}
%\DeclarePairedDelimiter{\floor}{\lfloor}{\rfloor}

\definecolor{shadethmcolor}{HTML}{F0F0F0}
%\definecolor{shadethmcolor}{HTML}{EDEDED}
%\definecolor{shadethmcolor}{HTML}{EDF8FF}
%\definecolor{shaderulecolor}{HTML}{EDF8FF}
%\definecolor{shaderulecolor}{HTML}{45CFFF}
\setlength{\shadeboxrule}{.4pt}

\setlength\parindent{0pt}

% Packages hyperref and algorithmic misbehave sometimes.  We can fix
% this with the following command.
\newcommand{\theHalgorithm}{\arabic{algorithm}}

%----------------------------------------
% Standard macros
%----------------------------------------


%----------------------------------------
% Project-specific macros
%----------------------------------------

%----------------------------------------
% Header
%----------------------------------------
\title{IFT 6085 - Lecture 3 \\ 
Gradients for smooth and for strongly convex functions}
\date{}
%----------------------------------------
% Document
%----------------------------------------
\begin{document} 

%----------------------------------------
% Abstract
%----------------------------------------
\maketitle

\vspace{-0.5in}
\begin{center}
This version of the notes has not yet been thoroughly checked.
Please report any bugs to the scribes or instructor.
\end{center}
\vspace{0.2in}

\textbf{Scribes}\hfill
\textbf{Instructor:}  Ioannis Mitliagkas\\
\textbf{Winter 2020:} Thong (Bob) Vo\\
\textbf{Winter 2019:} Amir Raza, Philippe Lacaille, Jonathan Plante\\
\textbf{Winter 2018:} Philippe Brouillard, Massimo and Lucas Caccia


%----------------------------------------
% Body
%----------------------------------------

\newcommand{\infgc}{\inf_{g \in \mathcal{C}}}
\newcommand{\supgc}{\sup_{g \in \mathcal{C}}}

\newcommand{\Prob}{\mathbb{P}}
\newcommand{\reals}{\mathbb{R}}

\section{Summary}

In the previous lecture we covered the notions of convexity as well as Lipschitsz continuity. 
After introducing these concepts, a bound on the convergence rate of gradient descent of a convex and $L$-Lipschitz function was demonstrated to scale with the $\sqrt{T}$; we proved that the upper bound on thee \emph{sub-optimality gap} was $\mathcal{O}\parens{\frac{1}{\sqrt{T}}}$.

In this lecture, we build on some of the notions from the previous lecture and introduce guarantees on the convergence rate of gradient descent for a stronger family of functions (using stronger assumptions), namely $\beta$-smooth and $\lambda$-strongly convex functions.

\section{Gradient Descent for smooth functions}
\label{sec:gradient}
\begin{defn}[$\beta$-smoothness]
We say that a continuously differentiable function $f$ is $\beta$-smooth if its gradient $\nabla f$ is $\beta$-Lipschitz, that is 
\[
	\norm{\nabla f(\vx) - \nabla f(\vy)} \leq \beta \norm{\vx-\vy}.
\]
\end{defn}
If we recall Lipschitz continuity from Lecture 2, simply speaking, an \textit{L-}Lipschitz function is limited by how \textit{quickly} its output can change. By imposing this same constraint on the gradients of a function, $\beta$-smoothness implies they cannot change abruptly and must be bounded by some value as defined above.
\\

% \begin{figure}[ht]
% \centering
%     \includegraphics[width=0.58\textwidth]{BSmoothPic.png}%
%     \caption{(a) A $\beta$-smooth function. (b) A non $\beta$-smooth function}
%     \label{str_convex}
% \end{figure}

In other other words, $\beta$-smoothness is putting an upper bound on the curvature of the function. 
This is equivalent to the eigenvalues of the Hessian being less than $\beta$. 
Note that there can be $\beta$-smooth functions which are not twice differentiable. 
One key benefit of these functions is that their gradients tend to decay when $\vx$ gets closer to the minimum. 
In contrast, non-smooth functions may have abrupt bends at the minimum, which cause significant oscillations for gradient descent. Figure \ref{fig:gradient_updates} illustrates this point by comparing the two scenarios.
This super-nice property of smooth functions results in the \emph{self-tuning property}; we do \emph{not} need to tune the step-size with updates, or pre-define it based on the total number of updates as we did in the last lecture, in order to make gradient descent work! 

\begin{figure}[ht]
\centering
    \includegraphics[width=0.87\textwidth]{gradient_swing_legends.png}%
    \caption{(a) A smooth function with decaying updates. (b) A non-smooth function with oscillating updates.}
    \label{fig:gradient_updates}
\end{figure}


\subsection{Convex and smooth functions}

Here we introduce a bound on the convergence rate of a convex and $\beta$-smooth function.

\begin{lemma}[Quadratic bounds]\label{lemma:smooth-sandwich}
Let $f$ be $\beta$-smooth on $\mathbb{R}^n$. Then for any $\vx,\vy \in \mathbb{R}^n$, one has
\begin{equation} \label{eq:lemma34}
	\abs{f(\vx) - f(\vy) - \nabla f(\vy)^\top (\vx-\vy)} \leq \frac{\beta}{2} \norm{\vx-\vy}^2.
\end{equation}
%See \cite{bubeck2015convex} Lemma 3.4 for proof.
\end{lemma}
\begin{proof} Let $g: \mathbb{R}\to \mathbb{R}$ be the function defined as: $g(t) = f(\vy+t(\vx-\vy))$. Then, the following holds true:
\[
    f(\vx)=g(1)  \text{ and }  f(\vy) = g(0).
\]

Then, from the fundamental theorem of calculus, we observe that
\begin{equation} \label{eq:gtintegral}
    f(\vx) - f(\vy) = g(1) - g(0) = \int_{0}^{1}g'(t) \, dt .
\end{equation}
Taking derivative of $g(t):$
\begin{equation} \label{eq:gtderivative}
    g'(t) = \nabla f(\vy+t(\vx-\vy))^\top (\vx-\vy).
\end{equation}
Plugging in Equation \ref{eq:gtintegral}, and \ref{eq:gtderivative} to the LHS of \ref{eq:lemma34}:
\begin{align*}
	& \abs{f(\vx) - f(\vy) - \nabla f(\vy)^\top (\vx-\vy)} &\\
	&= \abs{\int_{0}^{1} \nabla f(\vy+t(\vx-\vy))^\top (\vx-\vy) \, dt - \nabla f(\vy)^\top (\vx-\vy)} \\
	&\leq \int_{0}^{1} \abs{\nabla f(\vy+t(\vx-\vy)-\nabla f(\vy))^\top (\vx-\vy)} \, dt \\
	&\leq \int_{0}^{1} \norm{\nabla f(\vy+t(\vx-\vy))-\nabla f(\vy)} \cdot \norm{\vx-\vy}  \, dt \tag{applying Cauchy–Schwarz inequality  } \\
	&\leq \int_{0}^{1} \beta t\norm{\vx-\vy}^2 \, dt \tag{the gradient $\nabla f$ is $\beta$  -Lipschitz.} \\
	&= \frac{\beta}{2}\norm{\vx-\vy}^2
\end{align*}{}
\end{proof}
\noindent Intuitively, the quadratic bound from Lemma~\ref{lemma:smooth-sandwich} tells us by how much the linear approximation of a smooth function is allowed to go off!
Note that the expression, $f(\vx) - f(\vy) - \nabla f(\vy)^\top (\vx-\vy)$ is the error in estimating the value of $f$ at point $\vx$ using the linear approximation of $f$ at point $\vy$.
Thus, the Lemma states that the absolute error in this linear approximation is no larger than $\frac{\beta}{2}$ times the squared distance (i.e., $\norm{\vx-\vy}^2$) between the two points $\vx, \vy$.\\[5pt]
\noindent Now, in addition to $f$ being $\beta-$smooth, if we further assume that $f$ is convex, then we can use the first-order condition of convexity (Lemma 2 of the previous lecture) to establish an additional result that will turn out to be useful for proving the convergence guarantees for convex and smooth functions.

\begin{lemma}
\label{improvedBoundsForConvexAndSmoothFunctions}
Let $f$ be a convex and $\beta-$smooth function on $\mathbb{R}^n$. 
Then, $\forall\,\vx,\vy$ $\in \mathbb{R}^n$,  we have--
\begin{align}
    \text{\emph{\textbf{1.}}}\quad & 
    0 \leq f(\vx)-f(\vy) - \nabla f(\vy)^\top(\vx-\vy) \leq \dfrac{\beta}{2} \norm{\vx-\vy}^2\ \text{and further,}
    \nonumber\\
    \text{\emph{\textbf{2.}}}\quad &
    \frac{1}{2 \beta} \norm{\nabla f(\vx)-\nabla f(\vy)}^2
    \leq
	f(\vx) - f(\vy) - \nabla f(\vy)^\top (\vx-\vy)
	\nonumber
\end{align}

\end{lemma}
\begin{proof}
% \textit{See \cite{bubeck2015convex} Lemma 3.5 for proof.}
Statement \textbf{1.} follows from the quadratic bound and Lemma 2 of the previous lecture.\\[5pt]
To see how statement \textbf{2.} follows, set 
$
    \vz 
    = \vx - \frac{1}{\beta}\left(
        \nabla f\left(\vx\right) - \nabla f\left(\vy\right)
    \right)
$ and consider the following manipulations.
\begin{align*}
	f\left(\vy\right) - f\left(\vx\right)
	&=
	\underline{f\left(\vy\right) - f\left(\vz\right)}
	+
	f\left(\vz\right) - f\left(\vx\right)
	\nonumber\\
	&
	\leq 
	\nabla f\left(\vy\right)^\top\left(\vy - \vz\right)
	+
	\underline{
	    f\left(\vz\right) - f\left(\vx\right)
	 }\ 
	\tag{Lemma 2 of previous lecture}
	\nonumber\\
	&
	\leq 
	\underline{
	\nabla f\left(\vy\right)^\top\left(\vy - \vz\right)
	+
	\nabla f\left(\vx\right)^\top \left(\vz - \vx\right)
	}
	+ 
	\frac{\beta}{2}\left\lVert\vz - \vx\right\rVert^2
	\ 
	\tag{RHS of statement \textbf{1.}}
	\nonumber\\
	&
	=
	\nabla f\left(\vy\right)^\top\left(\vy - \vx\right)
	+
	\left(
	    \nabla f\left(\vy\right)
	    -
	    \nabla f\left(\vx\right)
	\right)^\top
	\underline{
	\left(\vx - \vz\right)
	}
	+ 
	\frac{\beta}{2}
	\left\lVert
	\underline{
	\vz - \vx
	}
	\right\rVert^2
	\nonumber\\
	&
	=
	\nabla f\left(\vy\right)^\top\left(\vy - \vx\right)
	+
	\underline{
	\frac{1}{\beta}
	\left(
	    \nabla f\left(\vy\right)
	    -
	    \nabla f\left(\vx\right)
	\right)^\top
	\left(
	    \nabla f\left(\vx\right) - \nabla f\left(\vy\right)
	\right)
	}
	\nonumber\\
	&
	\qquad\qquad
	+
	\underline{
	\frac{\beta}{2}
	\left\lVert
	-\frac{1}{\beta}\left(
	    \nabla f\left(\vx\right) - \nabla f\left(\vy\right)
	\right)
	\right\rVert^2
	}\ 
	\tag{substituting the expression for $\vz$}
	\nonumber\\
	&
	=
	\nabla f\left(\vy\right)^\top\left(\vy - \vx\right)
	-
	\frac{1}{2\beta}
	\left\lVert
	    \nabla f\left(\vx\right) - \nabla f\left(\vy\right)
	\right\rVert^2\ 
	\tag{rearranging and simplifying terms}
\end{align*}{}
Rearranging the terms of the inequality obtained from the above manipulations gives the desired result.
\end{proof}
\noindent Note that what this result tells us is the following.
Statement \textbf{1.} tells that the error in linear approximation as discussed above is lower bounded by 0 and upper bounded by $\frac{\beta}{2} \norm{\vx-\vy}^2$ when $f$ is convex in addition to being $\beta-$smooth.
However, statement \textbf{2.} tightens the lower bound by improving it from 0 to $\frac{1}{2 \beta} \norm{\nabla f(\vx)-\nabla f(\vy)}^2$.
Now, using this Lemma~\ref{improvedBoundsForConvexAndSmoothFunctions}, we can prove the convergence rate bound for convex and smooth functions.

\begin{thm}\label{thm:cvx-smth}
Let $f$ be convex and $\beta$-smooth on $\mathbb{R}^n$. Then the gradient descent with step size $\gamma = 1/\beta$ satisfies
\[
	f(\vx^{(k)}) - f(\vx^\ast) \leq \frac{2 \beta \norm{\vx^{(1)} - \vx^\ast}^2}{k-1}.
\]
\end{thm}

\begin{proof}
\textit{
    We skip the proof for the sake of brevity.
    Refer to \cite{bubeck2015convex} page 268 for the same.
}
\end{proof}

Notice that stronger assumptions on the function $f$ lead to improvements in the convergence rate.
In comparison to the convergence analysis for a convex and $L$-Lipschitz function, the following points of improvement are observed from Theorem~\ref{thm:cvx-smth} for a convex and $\beta-$smooth function:
\begin{itemize}
    \item No averaging of terms is needed; the guarantee holds for $f(\vx^{(k)})$, the function value at the last iterate $\vx^{(k)}$.
    \item The last iterate $\vx^{(k)}$ is a good solution; there is no need to keep track of previous iterations.
    \item Convergence scales linearly with number of steps; we have a convergence rate of $\mathcal{O}(1/T)$ instead of $\mathcal{O}(1/\sqrt{T})$.
    \item Ideal step size $\gamma$ is constant and does not depend on $T$, number of steps taken.
\end{itemize}
However, the convergence rate bound still has the term ${\norm{\vx^{(1)} - \vx^\ast}^2}$ and thus, the convergence rate depends on the choice of initialization $\vx^{(1)$. 
In particular, the bound will be tighter if $\vx^{(1)}$ is closer to $\vx^\ast$ (the minimum), and looser if $\vx^{(1)}$ is farther. 
Bounds that don't depend on the initial value of $\vx$ will be discussed later in this lecture.

\section{Strong convexity}

\begin{defn}[Strong convexity]
A function $f(\vx)$ is $\lambda$-strongly convex, if for $\lambda>0$, $\forall \vx \in \dom(f)$, 
\begin{align*}
    f(\vx) - \frac{\lambda}{2} \norm{\vx}^2 \text{ is convex.}
\end{align*}
\end{defn}

Strong convexity provides a lower bound for the function's curvature. The function must have strictly positive curvature. In other words, all eigenvalues of the Hessian of a $\lambda$-strongly convex function are lower-bounded by $\lambda$. We can write this in terms of positive semi-definiteness as
\[
    \nabla^2 f(\vx) \succeq \lambda \mI \iff \nabla^2 f(\vx) - \lambda \mI \succeq 0.
\]

For example, $f:\R\to\R, f(x) = \frac{h}{2} x^2$ is $h$-strongly convex, but not $(h+\epsilon)$-strongly convex for $\epsilon>0$. Figure~\ref{str_convex} illustrates examples of two convex functions, of which only one is strongly convex.\\

\begin{figure}[ht]
\centering
    \includegraphics[width=0.58\textwidth]{str_convex.png}%
    \caption{(a) A convex function which is also strongly convex. (b) A convex function which is not strongly convex.}
    \label{str_convex}
\end{figure}

\begin{lemma}\label{lemma:str-cvx-bound}
Let $f$ be $\lambda$-strongly convex. Then $\forall \vx, \vy \in \dom(f)$, we have
\[
    f(\vy) - f(\vx) \leq \nabla f(\vy)^\top (\vy-\vx) - \frac{\lambda}{2} \norm{\vx - \vy}^2.
\]
\end{lemma}
\begin{proof}
The proof follows from the definition of strong convexity and the linear approximation idea for convex functions as seen before.
Try proving it for yourself!
\end{proof}

\subsection{Strongly convex and Lipschitz functions}
\begin{thm}\label{thm:str-cvx-lipschitz}
Let $f$ be $\lambda$-strongly convex and $L$-Lipschitz. Then the projected subgradient descent after $T$ steps with $\gamma_k = \frac{2}{\lambda(k+1)}$ satisfies  
\[
	f\parens{\sum_{k=1}^T \frac{2k}{T(T+1)} \vx^{(k)}} - f(\vx^\ast) \leq \frac{2 L^2}{\lambda(T+1)}.
\]
\end{thm}

\begin{proof}
We start with Lemma \ref{lemma:str-cvx-bound}:
\begin{align*}
    f(\vx^{(k)}) - f(\vx^\ast) &\leq \nabla f(\vx^{(k)})^\top (\vx^{(k)} - \vx^\ast) - \frac{\lambda}{2} \norm{\vx^{(k)} - \vx^\ast}^2 \\
    &= \frac{1}{\gamma_k} (\vx^{(k)} - \vx^{(k+1)})^\top (\vx^{(k)} - \vx^\ast) - \frac{\lambda}{2} \norm{\vx^{(k)} - \vx^\ast}^2 \\
    &= \frac{1}{2\gamma_k} \parens{\norm{\vx^{(k)} - \vx^{(k+1)}}^2 + \norm{\vx^{(k)} - \vx^\ast}^2 - \norm{\vx^{(k+1)} - \vx^\ast}^2} - \frac{\lambda}{2} \norm{\vx^{(k)} - \vx^\ast}^2 \tag{$2 \va^\top \vb = \norm{\va}^2 + \norm{\vb}^2 - \norm{\va - \vb}^2$} \\
    &= \frac{\gamma_k}{2} \norm{\nabla f(\vx^{(k)})}^2 + \frac{1}{2\gamma_k} \parens{\norm{\vx^{(k)} - \vx^\ast}^2 - \norm{\vx^{(k+1)} - \vx^\ast}^2} - \frac{\lambda}{2} \norm{\vx^{(k)} - \vx^\ast}^2 \\
    &\leq \frac{\gamma_k}{2} L^2 + \parens{\frac{1}{2\gamma_k} - \frac{\lambda}{2}} \norm{\vx^{(k)} - \vx^\ast}^2 - \frac{1}{2\gamma_k} \norm{\vx^{(k+1)} - \vx^\ast}^2 \tag{$f$ is $L$-Lipschitz}
\end{align*}

We can now multiply both sides by $k$ and sum from 1 to $T$:
\begin{align*}
    \sum_{k=1}^{T} k \parens{f(\vx^{(k)}) - f(\vx^\ast)} &\leq \sum_{k=1}^{T} k \parens{\frac{\gamma_k}{2} L^2 + \parens{\frac{1}{2\gamma_k} - \frac{\lambda}{2}} \norm{\vx^{(k)} - \vx^\ast}^2 - \frac{1}{2\gamma_k} \norm{\vx^{(k+1)} - \vx^\ast}^2} \\
    &= \sum_{k=1}^{T} \parens{\frac{k L^2}{\lambda (k+1)} + k\parens{\frac{\lambda (k+1)}{4} - \frac{\lambda}{2}} \norm{\vx^{(k)} - \vx^\ast}^2 - \frac{\lambda k(k+1)}{4} \norm{\vx^{(k+1)} - \vx^\ast}^2} \tag{substituting $\gamma_k = \frac{2}{\lambda(k+1)}$} \\
    &= \sum_{k=1}^{T} \parens{\frac{k L^2}{\lambda (k+1)} + \frac{\lambda}{4} \parens{k(k-1)\norm{\vx^{(k)} - \vx^\ast}^2 - k(k+1)\norm{\vx^{(k+1)} - \vx^\ast}^2}} \\
    &\leq \sum_{k=1}^{T} \parens{\frac{L^2}{\lambda} + \frac{\lambda}{4} \parens{k(k-1)\norm{\vx^{(k)} - \vx^\ast}^2 - k(k+1)\norm{\vx^{(k+1)} - \vx^\ast}^2}} \tag{$\frac{k}{k+1} \leq 1$} \\
    &= \frac{T L^2}{\lambda} + \underbrace{\sum_{k=1}^{T} \parens{\frac{\lambda}{4} \parens{k(k-1)\norm{\vx^{(k)} - \vx^\ast}^2 - k(k+1)\norm{\vx^{(k+1)} - \vx^\ast}^2}}}_{\leq 0} \\
    &\leq \frac{T L^2}{\lambda}
\end{align*}

We can now apply Jensen's inequality to the left-hand side. Note that for a convex function $\phi$ Jensen's inequality tells us that
\[
    \phi\parens{\frac{\sum_i a_i x_i}{\sum_i a_i}} \leq \frac{\sum_i a_i \phi(x_i)}{\sum_i a_i}.
\]

In our case here we have
\begin{gather*}
    f\parens{\frac{\sum_{k=1}^{T} k \vx^{(k)}}{\sum_{k=1}^{T} k}} \leq \frac{\sum_{k=1}^{T} k f(\vx^{(k)})}{\sum_{k=1}^{T} k} \iff \parens{\sum_{k=1}^{T} k} f\parens{\frac{\sum_{k=1}^{T} k \vx^{(k)}}{\sum_{k=1}^{T} k}} \leq \sum_{k=1}^{T} k f(\vx^{(k)}).
\end{gather*}

Since $\sum_{k=1}^{T} k = \frac{T(T+1)}{2}$, we have
\begin{align*}
    \sum_{k=1}^{T} k \parens{f(\vx^{(k)}) - f(\vx^\ast)} &= \sum_{k=1}^{T} k f(\vx^{(k)}) - f(\vx^\ast) \parens{\sum_{k=1}^{T} k} \\
    &\geq \parens{\sum_{k=1}^{T} k} f\parens{\sum_{k=1}^{T} \frac{2k}{T(T+1)} \vx^{(k)}} - f(\vx^\ast) \parens{\sum_{k=1}^{T} k}
\end{align*}

Coming back to our previous inequality, we have:
\[
    \parens{\sum_{k=1}^{T} k} f\parens{\sum_{k=1}^{T} \frac{2k}{T(T+1)} \vx^{(k)}} - f(\vx^\ast) \parens{\sum_{k=1}^{T} k} \leq \frac{T L^2}{\lambda}
\]

We use the fact that $\sum_{k=1}^{T} k = \frac{T(T+1)}{2}$ once again to complete the proof:
\[
    f\parens{\sum_{k=1}^{T} \frac{2k}{T(T+1)} \vx^{(k)}} - f(\vx^\ast) \le \frac{2 L^2}{\lambda (T+1)}
\]
\end{proof}

With Theorem \ref{thm:str-cvx-lipschitz}, we can notice how moving from convexity to strong convexity may affect the convergence rate of gradient descent. We previously tackled convexity paired with Lipschitz continuity in Lecture 2 and the similarity with the new convergence guarantees is pretty noticeable. By moving to strong convexity, we can notice a few things:
\begin{itemize}
    \item We still have averaging of terms in the guarantee. 
    However, the averaging is \emph{weighted}; $\sum_{k=1}^T \frac{2k}{T(T+1)} \vx^{(k)}$ is a non-uniform averaging scheme with more recent iterates having higher weight, which is intuitive.
    \item The final iterate $\vx^{(k)}$ is not appropriate.
    Ideally, we would only want to have $f$ evaluated at $\vx^{(k)}$ in the guarantee.
    \item Convergence still scales linearly with number of steps.
    \item Although not constant, the step size $\gamma_k$ is diminishing at every step
\end{itemize}

Unlike previous results, the right hand side does not have any terms dependent on distance from $\vx^\ast$ here. Intuitively this is a consequence of two conditions which have to be met for a strongly convex function:
\begin{itemize}
    \item Norm of gradient increases as we go further away from minimum.
    \item Gradient for a $L$-Lipschitz function is bounded, and can not keep increasing.
\end{itemize}
Thus, intuitively, there is no distance term on the right hand side of the bound to have a finite bound.
One could hope that by combining strong convexity along with smoothness, gradient descent may present stronger convergence guarantees. 
We will see how smoothness removes dependency from the averaging scheme.


Note: $\lambda$-strongly convex and $L$-Lipschitz condition is a special case because the upper bound $L$-Lipschitz condition will ultimately conflict with the lower bound $\lambda$-strongly convex growth rate. Therefore, such functions are typically considered only on a range, e.g. $x \in [-1,1]$.


\subsection{Strongly convex and smooth functions}
Lemma \ref{lemma:smooth-sandwich} (quadratic bounds) gives us that a $\beta$-smooth function $f$ is sandwiched between two quadratics because of the following inequality:

\begin{equation}\label{eq:smooth-bound}
    f(\vy) + \nabla f(\vy)^\top (\vx-\vy) - \frac{\beta}{2} \norm{\vx-\vy}^2 \leq f(\vx) \leq f(\vy) + \nabla f(\vy)^\top (\vx-\vy) + \frac{\beta}{2} \norm{\vx-\vy}^2
\end{equation}

Similarly, by rearranging terms in Lemma \ref{lemma:str-cvx-bound} tells us that a $\lambda$-strong convex function can be lower-bounded by the following inequality:
\begin{equation}\label{eq:str-cvx-bound}
    f(\vx) \geq f(\vy) - \nabla f(\vy)^\top (\vy-\vx) + \frac{\lambda}{2} \norm{\vx - \vy}^2.
\end{equation}

Figure \ref{fig:bounds} showcases the resulting bounds from both the smoothness and the strong convexity constraints. The shaded area in each sub-figure is showing the area validated by the respective bound(s).

\begin{figure}[ht]
    \includegraphics[width=.32\textwidth]{smooth-bound.png}\hfill
    \includegraphics[width=.32\textwidth]{str-cvx-bound.png}\hfill
    \includegraphics[width=.32\textwidth]{smooth-str-cvx-bound.png}

    \caption{(Left) Upper and lower bounds from equation \ref{eq:smooth-bound} for smoothness constraint (Middle) Lower bound from equation \ref{eq:str-cvx-bound} for strong convexity constraint (Right) Combination of upper bound from smoothness and lower bound from strong convexity}
    \label{fig:bounds}
\end{figure}

\begin{figure}[h]
\centering
    \includegraphics[width=0.4\textwidth]{lipschitz-bound.png}
    \caption{For an $L_f$-Lipschitz continuous function, the green region shows where the function would exist. We can imagine that without smoothness and only $L$-Lipschitz in equation \ref{eq:smooth-bound}, the accepted region would be having linear boundaries}
    \label{fig:accepted-reg}
\end{figure}


\begin{lemma}[Coercivity of the gradient] Let $f$ be $\beta$-smooth and $\lambda$-strongly convex. Then for all $x$ and $y$, we have:
\[
    \inner{\nabla f(\vx) - \nabla f(\vy), \vx-\vy} \geq \frac{\lambda\beta}{\lambda + \beta} \norm{\vx-\vy}^2 + \frac{1}{\lambda + \beta} \norm{\nabla f(\vx) - \nabla f(\vy)}^2.
\]
\end{lemma}

\begin{proof}
\textit{See \cite{bubeck2015convex} Lemma 3.11 for proof, which we skip for brevity.}
\end{proof}

\begin{thm}\label{thm:str-cvx-smth}
For $f$ a $\lambda$-strongly convex and $\beta$-smooth function, gradient descent with $\gamma = \frac{2}{\lambda + \beta}$ satisfies:
\[
	f(\vx^{(k+1)}) - f(\vx^\ast) \leq \frac{\beta}{2} \exp\parens{ -\frac{4k}{\kappa + 1}} \norm{\vx^{(1)} - \vx^\ast}^2
\]
where $\kappa := \frac{\beta}{\lambda}$ is the condition number.
\end{thm}

\begin{proof}
First, let's define the distance between the coordinate at the iteration $k$ and the optimal point as:
\[
    D_k := \norm{\vx^{(k)} - \vx^\ast}.
\]

By replacing $\vx^{(k+1)}$ by its definition $\vx^{(k)} - \gamma \nabla f(\vx^{(k)})$, we get:
\[
    D_{k+1}^2 = \norm{\vx^{(k)} - \gamma \nabla f(\vx^{(k)}) - \vx^\ast}^{2}.
\]

By expanding the square, we get:
\[
    D_{k+1}^2 = \norm{\vx^{(k)} - \vx^\ast}^2 - 2 \gamma \inner{\nabla f(\vx^{(k)}), \vx^{(k)} - \vx^\ast} + \gamma^2 \norm{\nabla f(\vx^{(k)})}^2.
\]

By doing a slight modification to the second term, we can apply the lemma of coercivity of the gradient. Since $ \nabla f(\vx^\ast) = \vzero$ this equality holds:
\[
    \langle \nabla f(\vx^{(k)}), \vx^{(k)} - \vx^\ast \rangle = \langle \nabla f(\vx^{(k)}) - \nabla f(\vx^\ast), \vx^{(k)} - \vx^\ast \rangle.
\]

By applying the lemma of coercivity of the gradient, we get an upper bound:
\[
    \inner{\nabla f(\vx^{(k)}) - \nabla f(\vx^\ast), \vx^{(k)} - \vx^\ast} \geq \frac{\lambda \beta}{\lambda + \beta} D^2_k + \frac{1}{\lambda + \beta} \norm{\nabla f(\vx^{(k)}) - \nabla f(\vx^\ast)}^2.
\]

By replacing the second term by the upper bound we just found, we get:
\[
    D^2_{k+1} \leq D^2_k - 2 \gamma \parens{\frac{\lambda \beta}{\lambda + \beta} D^2_k + \frac{1}{\lambda + \beta} \norm{\nabla f(\vx^{(k)})}^2} + \gamma^2 \norm{\nabla f(\vx^{(k)})}^2.
\]

We can rearrange the terms and add $-\nabla f(\vx^\ast)$ again inside the norm terms:
\[
    D_{k+1}^2 \leq \parens{1 - \frac{2\gamma \lambda \beta}{\lambda + \beta}} D^2_k + \parens{\frac{-2\gamma}{\lambda + \beta} + \gamma^2} \norm{\nabla f(\vx^{(k)}) - \nabla f(\vx^\ast)}^2.
\]

The term $\left(1-\frac{2\gamma \lambda \beta}{\lambda + \beta}\right)$ is useful, because we can show that it is less than 1 and has geometric convergence. Further steps will try to simplify the remaining terms.

We can change the norm term by using the fact that $f$ is $\beta$-smooth:
\[
    D_{k+1}^2 \leq \parens{1 - \frac{2\gamma \lambda \beta}{\lambda + \beta}} D^2_k + \parens{\frac{-2\gamma}{\lambda + \beta} + \gamma^2} \beta^2 D^2_k.
\]

Let $\gamma = \frac{2}{\lambda + \beta}$, then:
\[
    D^2_{k+1} \leq \parens{1 - \frac{4 \lambda \beta}{(\lambda + \beta)^2}} D^2_k.
\]

By unrolling the recursion and since $\parens{\frac{\kappa - 1}{\kappa + 1}}^2 = \parens{1 - \frac{4 \lambda \beta}{(\lambda + \beta)^2}}$, we get:
\[
    D^2_{k+1} \leq \parens{\frac{\kappa - 1}{\kappa + 1}}^{2k} D_1^2.
\]

Since $\exp(-u) \geq 1-u$ for every $u$, we get:
\[
    D_{k+1}^2 \leq \exp{\parens{-\frac{4k}{\kappa + 1}}} D_1^2.
\]

By $\beta$-smoothness we finally have:
\[
    f(\vx^{(k+1)}) - f(\vx^\ast) \leq \frac{\beta}{2} \exp{\parens{-\frac{4k}{\kappa + 1}}} \norm{\vx^{(1)} - \vx^\ast}^2.
\]
\end{proof}

Theorem \ref{thm:str-cvx-smth} observations
\begin{itemize}
    \item $\vx^{(k)}$ solution is a good solution, i.e. no reference to previous iterations in the guarantee.
    \item We achieve a fast convergence rate of $\mathcal{O}(\exp(-T))$.
    \item $\kappa$ measures how \emph{far apart} the upper and lower bounds are (see Figure \ref{fig:bounds}). 
    It can be interpreted as the ratio of largest to smallest curvature of the function.
    \item The smaller the condition number $\kappa$ is, the less iterations are required to converge. Intuitively, the accepted region between the bounds will be smaller.
    \item Consequently, the greater $\beta$ is the more iterations will be required to converge. This is logical since a constant step size on a function with a steep gradient will cause the a greater change in the function value.
\end{itemize}

\section{Comparison of optimization properties for different function classes}
Table \ref{tab:comparison} summarizes various convergence properties of discussed functions classes.  From top to bottom, the assumptions on properties of these classes increase, from convex and $L$-Lipschitz to $\lambda$-strongly convex and $\beta$-smooth. In the same direction, the convergence rates are also increase, aligned with stronger assumptions on those functions.

\section*{Suggested exercises}
\begin{itemize}
    \item \citet{Shalev-Shwartz:2014:UML:2621980} -- \textbf{Chapter 12:} 2; \textbf{Chapter 14:} 1
    \item \citet{Gower} -- \textbf{Ex. 1:} P5, P6; \textbf{Ex. 2:} P3, P4, P5; \textbf{Ex. 3:} P1-4
\end{itemize}

\begin{landscape}
\begin{table}[p]
\centering
\caption{Convergence rates for different function classes in a nutshell}
\label{tab:comparison}
\begin{tabular}{@{}cccccc@{}}
\toprule
Function class & Optimal step-size & Convergence rate & Sub-optimal gap & & Bounds of the gap \\ \midrule
convex, $L$-Lipschitz &
    $\displaystyle \gamma = \frac{\norm{\vx^{(1)} - \vx^\ast}_2}{L \sqrt{T}}$ &
    $\displaystyle \mathcal{O}(1/\sqrt{T})$ &
    $\displaystyle f\parens{\frac{1}{T}\sum_{k=1}^T \vx^{(k)}} - f(\vx^\ast)$ &
    $\leq$ &
    $\displaystyle \frac{L}{\sqrt{T}} \norm{\vx^{(1)} - \vx^\ast}$ \\
convex, $\beta$-smooth &
    $\displaystyle \gamma = \frac{1}{\beta}$ &
    $\displaystyle \mathcal{O}(1/{T})$ &
    $\displaystyle f(\vx^{(k)}) - f(\vx^\ast)$ &
    $\leq$ &
    $\displaystyle \frac{2 \beta}{k-1} \norm{\vx^{(1)} - \vx^\ast}^2$ \\
$\lambda$-strongly convex, $L$-Lipschitz &
    $\displaystyle \gamma = \frac{2}{\lambda (k+1)}$ &
    $\displaystyle \mathcal{O}(1/{T})$ &
    $\displaystyle f\parens{\sum_{k=1}^T \frac{2k}{T(T+1)} \vx^{(k)}} - f(\vx^\ast)$ &
    $\leq$ &
    $\displaystyle \frac{2 L^2}{\lambda(T+1)}$ \\
$\lambda$-strongly convex, $\beta$-smooth &
    $\displaystyle \gamma = \frac{2}{\lambda + \beta}$ &
    $\displaystyle \mathcal{O}(\exp(-T))$ &
    $\displaystyle f(\vx^{(k+1)}) - f(\vx^\ast)$ &
    $\leq$ &
    $\displaystyle \frac{\beta}{2} \exp \parens{-\frac{4k}{\kappa + 1}} \norm{\vx^{(1)} - \vx^\ast}^2$ \\
\bottomrule
\end{tabular}
\end{table}
\end{landscape}


%----------------------------------------
% \section*{Acknowledgments} 

%----------------------------------------
\bibliographystyle{abbrvnat}
\bibliography{Refs/lec3}
%----------------------------------------
\end{document}
